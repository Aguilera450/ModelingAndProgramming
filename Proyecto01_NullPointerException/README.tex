\documentclass{article}

% Symbols
\usepackage{amsfonts, amsthm}
\usepackage{upgreek}
\usepackage{physics}
\usepackage{cancel}
\usepackage{amssymb, latexsym, amsmath}
\usepackage{import}

%Algorithms
\usepackage[ruled,lined,linesnumbered,commentsnumbered]{algorithm2e}

%% Identación
\setlength{\parindent}{0cm}

% Código
\newcommand{\code}[1]{\textcolor{white!25!black}{\texttt{#1}}}
\usepackage{listings}

%AMS
\usepackage{amsthm}
\newtheorem{algo-thm}{Algoritmo}

% Graphics
\usepackage{graphicx}
\usepackage{pgf}

% Margins
\addtolength{\voffset}{-1.5cm}
\addtolength{\hoffset}{-1.5cm}
\addtolength{\textwidth}{3cm}
\addtolength{\textheight}{3cm}

%Header-Footer
\usepackage{fancyhdr}
\renewcommand{\headrulewidth}{1pt}

\newcommand{\set}[1]{
  \left\{ #1 \right\}
}

\footskip = 50pt
\renewcommand{\headrulewidth}{1pt}

\pagestyle{fancyplain}

\begin{document}
\title{UNIVERSIDAD NACIONAL AUT\'ONOMA DE M\'EXICO\\ Facultad de Ciencias}
\author{Equipo NullPointerException:\\
  Diego Angel Rosas Franco   - 318165330 \\
  Adri\'an Aguilera Moreno   - 421005200\\
  Marco Antonio Rivera Silva - 318183583}
\date{}
\maketitle
\begin{center}
  \includegraphics[scale=0.20]{../Portada/Portada}\\[0.4cm]
  \Large
  \bf{Modelado y programación}
  \normalsize
\end{center}
\newpage
\fancyhead[r]{ Modelado y programación 2022-2}


\section*{\LARGE{Proyecto 01}}

\subsection*{Instrucciones de instalación, compilación y ejecución.}
Se dará por hecho que el usuario sabe moverse en terminal.
\subsubsection*{Requerimientos previos:}
\begin{itemize}
\item[-] Se debe contar con una versión de Java $8$ o superior en su computadora. De preferencia la versión más reciente.
\end{itemize}

\subsubsection*{Ejecución del proyecto:}
\begin{itemize}
\item[-] Si está leyendo esto significa que desempaquetó con éxito el proyecto.
\item[-] Abra su terminal y diríjase a la ruta donde desempaquetó el proyecto.
\item[-] Una vez estando en la ruta
  
  \code{Proyecto01\_NullPointerException},
  
  diríjase a

  \code{Proyecto01\_NullPointerException/src/fciencias/modelado/}
\item[-] Ejecute: “\code{javac CheemsMart.java}”, esto generará los .class del proyecto.
\item[-] Ejecute: “\code{java CheemsMart}”, esto ejecutará el proyecto mostrándole el menú solicitado para la practica.
\end{itemize}
\newpage
\subsection*{Justificación de uso de patrones:}
\subsubsection*{Strategy}
Utilizamos Strategy en en la parte de cambiar el idioma debido a que lo resuelve
de una forma conveniente ya que teníamos pensado utilizar State, sin embargo notamos
que a pesar que en ambos tendríamos que modificar el idioma en tiempo de ejecución
había una gran desventaja al usar State, por ejemplo limitar los estados con múltiples
condiciones para evitar que se tuviera más de un estado a la vez, en cambio Strategy
podíamos controlar que en efecto solo tuviera un idioma de una manera mucho más sencilla
y efectiva, además de que sería sumamente sencillo agregar nuevos idiomas en Strategy
que en State.
\subsubsection*{Decorator}
Implementamos el patrón de diseño Decorator en la parte de descuentos, pues los productos
pueden aplicar a uno o más descuentos, incluso del mismo tipo (al menos no hay una restricción
para esto) de manera conveniente en nuestra implementación, donde el \code{Producto} original
sin modificaciones en precios es el centro del decorador.

\subsection*{Otros patrones que se pudieron usar:}
\subsubsection*{Singleton}
Nos percatamos que podíamos haber usado Singleton en la clase Servidor, pues de esa forma
tendríamos un servidor centralizado en el que tendríamos a todos nuestros usarios y el catálogo
de la tienda. Sin embargo, vimos que esto podría causar un problema de rigidez, pues a futuro
es posible que la tienda planeé tener servidores a lo largo del mundo para mejorar la velocidad
de respuesta acorde a la región en la que se encuentre el usuario (teoricamente hablando), y
en cada servidor, regional por ejemplo, tener un catálogo distinto a futuro, ya que actualmente
solo se tiene un catálogo en el mismo idioma para todas las regiones. Debido a esto es que
Servidor cuenta con un identificador único, para asi poder diferenciarlo de los demás servidores
futuros.

\newpage
\subsection*{Diagrama UML:}
%\begin{center}
%  \includegraphics[scale=0.18]{./Practica04UML.png}
%\end{center}

\end{document}

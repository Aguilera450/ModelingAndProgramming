\documentclass{article}

% Symbols
\usepackage{amsfonts, amsthm}
\usepackage{upgreek}
\usepackage{physics}
\usepackage{cancel}
\usepackage{amssymb, latexsym, amsmath}
\usepackage{import}

%Algorithms
\usepackage[ruled,lined,linesnumbered,commentsnumbered]{algorithm2e}

%% Identación
\setlength{\parindent}{0cm}

% Código
\newcommand{\code}[1]{\textcolor{white!25!black}{\texttt{#1}}}
\usepackage{listings}

%AMS
\usepackage{amsthm}
\newtheorem{algo-thm}{Algoritmo}

% Graphics
\usepackage{graphicx}
\usepackage{pgf}

% Margins
\addtolength{\voffset}{-1.5cm}
\addtolength{\hoffset}{-1.5cm}
\addtolength{\textwidth}{3cm}
\addtolength{\textheight}{3cm}

%Header-Footer
\usepackage{fancyhdr}
\renewcommand{\headrulewidth}{1pt}

\newcommand{\set}[1]{
  \left\{ #1 \right\}
}

\footskip = 50pt
\renewcommand{\headrulewidth}{1pt}

\pagestyle{fancyplain}

\begin{document}
\title{UNIVERSIDAD NACIONAL AUT\'ONOMA DE M\'EXICO\\ Facultad de Ciencias}
\author{Equipo NullPointerException:\\
  Diego Angel Rosas Franco   - 318165330 \\
  Adri\'an Aguilera Moreno   - 421005200\\
  Marco Antonio Rivera Silva - 318183583}
\date{}
\maketitle
\begin{center}
  \includegraphics[scale=0.20]{../Portada/Portada}\\[0.4cm]
  \Large
  \bf{Modelado y programación}
  \normalsize
\end{center}
\newpage
\fancyhead[r]{ Modelado y programación 2022-2}


\section*{\LARGE{Práctica 3}}
Menciona los principios de diseño esenciales de los patrónes Decorator y Adapter:

\subsubsection*{Decorator}

Principios de diseño esenciales:
\newcommand{\localtextbulletone}{\textcolor{black}{\raisebox{.45ex}{\rule{.6ex}{.6ex}}}}
\renewcommand{\labelitemi}{\localtextbulletone}
\begin{itemize}
\item Hace que la composición de un objeto se pueda hacer de manera dinámica en tiempo de ejecución.
\item Es un patrón abierto al cambio y cerrado a la modificación.
\end{itemize}

Desventajas del patrón de diseño:
\begin{itemize}
\item Si quisieramos realizar modificaciones internas a las decoraciones más superficiales
  (\textit{e.g.}, modificaciones al núcleo) tenemos que ``desmontar'' las decoraciones hechas
  posteriormente a la que queremos modificar, esto le aporta rígidez a nuestro código.
\item El diseño se vuelve más complejo en términos de líneas de código.
\end{itemize}

\subsubsection*{Adapter}

Principios de diseño esenciales:
\begin{itemize}
\item Vuelve compatibles dos sistemas que antes no lo eran.
\item El usuario no percive que hay dos sistemas distintos funcionando como uno a tráves del Adapter.
\end{itemize}

Desventajas del patrón de diseño:
\begin{itemize}
\item Este patrón no optimiza lo que ya existe, solo parcha lo que hay y lo hace funcional (de funcionar),
  aunque esto no necesariamente implica que sea la mejor implementación.
\item Añade complejidad al diseño planteado inicialmente.
\end{itemize}


\subsection*{Instrucciones de instalación, compilación y ejecución.}
Se dará por hecho que el usuario sabe moverse en terminal.\\

\subsubsection*{Requerimientos previos:}
\begin{itemize}
\item[-] Se debe contar con Java en su computadora. De preferencia la versión más reciente.
\end{itemize}

\subsubsection*{Ejecución del proyecto:}
\begin{itemize}
\item[-] Si está leyendo esto significa que desempaquetó con éxito el proyecto.
\item[-] Abra su terminal y diríjase a la ruta donde desempaquetó el proyecto.
\item[-] Una vez estando en la ruta \code{Practica03\_NullPointerException}, diríjase a

  \code{Practica03\_NullPointerException/src/fciencias/modelado/}
\item[-] Ejecute: “javac WaySubXLasPizzasDeDonCangrejo.java”, esto generará los .class del proyecto.
\item[-] Ejecute: “java WaySubXLasPizzasDeDonCangrejo”, esto ejecutará el proyecto mostrándole el menú solicitado para la practica.
\end{itemize}

\newpage
\subsection*{Diagrama UML:}
\begin{center}
  \includegraphics[scale=0.16]{./Practica03UML.png}
\end{center}
\end{document}
